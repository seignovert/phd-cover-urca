%!TeX encoding = UTF-8

\firstname{Benoît}
\lastname{Seignovert}

\univ{de Reims Champagne-Ardenne}
\ecoledoct{Sciences, Technologie et Santé}

\logos{logo_ed}{}{logo_univ}
\discipline{Physique}
\specialite{Astrophysique}
\soutenue{publiquement}
\date{26 Septembre 2017}

\title{Analyse de la couche détachée de Titan\\à l'aide de l'instrument Cassini/ISS}
\titlemeta{Analyse de la couche détachée de Titan à l'aide de l'instrument Cassini/ISS}

\titleen{Analysis of Titan's Detached Haze Layer\\ with Cassini/ISS instrument}
\titlemetaen{Analysis of Titan's Detached Haze Layer with Cassini/ISS instrument}

\jury{ % Pas de président avant soutenance
M.       & Christophe \textsc{Sotin}     & Professeur & JPL & Président\\
M\up{me} & Caitlin    \textsc{Griffith}  & Professeure & UA-LPL & Rapportrice \\
M.       & Sébastien  \textsc{Lebonnois} & Directeur de Recherche & UPMC-LMD & Rapporteur \\
M\up{me} & Sandrine   \textsc{Vinatier}  & Chargée de Recherche & OBSPM-LESIA & Examinatrice \\
M.       & Pascal     \textsc{Rannou}    & Professeur & URCA-GSMA & Directeur de thèse \\
M.       & Panayotis  \textsc{Lavvas}    & Chargé de Recherche & URCA-GSMA & Co-directeur de thèse \\
}

\resume{
Chargée de brume photochimique, l'atmosphère de Titan est le siège d'une activité dynamique complexe évoluant lentement au gré des saisons. Tout comme la couche d'ozone sur Terre, la couche détachée sur Titan est une fine bande d'aérosols englobant la partie la plus externe de son atmosphère. Depuis sa découverte par les sondes Voyager dans les années 80, elle attise les curiosités quant à sa composition et ses mécanismes de formation.
Grâce à la sonde Cassini, en orbite autour de Saturne depuis 2004, nous avons une opportunité unique de l'observer sur près d'une demie année titanienne. Cette thèse vise à réaliser un suivi détaillé de la couche détachée sur l'ensemble de cette mission en s'appuyant sur les relevés réalisés par l'instrument ISS.
Dans un premier temps, nous présentons la procédure de traitement mise en place pour calibrer et géo-référencer les données du PDS.
Puis, nous réalisons une caractérisation des propriétés optiques des aérosols présents dans la couche détachée en couplant un modèle de diffusion par des agrégats fractals avec un modèle de transfert radiatif simplifié.
Par la suite, ces nouvelles contraintes sont réutilisées pour réaliser une inversion globale des profils d'extinction de la brume en s'appuyant sur un modèle plus complexe de transfert radiatif au limbe. Ces relevés systématiques nous permettent de suivre l'évolution spatiale et temporelle de la couche détachée tout au long de la mission Cassini.
Enfin, nous nous penchons tout particulièrement sur la disparition de la couche détachée au passage de l'équinoxe de printemps, suivi de sa réapparition en 2016, peu de temps avant le solstice d'été.
}
\motscles{Titan, Atmosphère, Aérosols, Transfert radiatif, Télédétection}

\abstract{
Loaded with photochemical haze, Titan's dynamical atmosphere is slowly evolving through the seasons. Like the ozone layer on Earth, Titan's detached haze layer is a thin coat of aerosols surrounding the upper part of its atmosphere. Since its discovery in the 80's by Voyagers' flybys, it has raised many questions on its content and origin. Thanks to the Cassini mission orbiting around Saturn since 2004, we have the chance to track it over half a Titan year. This thesis carries out a complet survey on Titan's detached haze layer observations taken continuously by the ISS instrument during the whole mission.
At first, we present the processing pipeline developed to calibrate and navigate the raw data coming from the PDS.
Then, we characterize the aerosols optical properties seen inside the detached haze layer by coupling a fractal aggregate model with a simplified version of the radiative transfer equation.
Thereafter, these new constrains are used as input into a more complex radiative transfer model in the limb geometry in order to extract globally the extinction profiles of the haze in the upper atmosphere. These systematic surveys allow us to follow the spatial and temporal evolution of the detached haze layer from the beginning to the end of the Cassini mission.
Finally, we took a special care on the disappearance of the detached haze after the vernal equinox and its recent reappearance in 2016, just before the summer solstice.
}
\keywords{Titan, Atmosphere, Aerosols, Radiative transfer, Remote Sensing}

\lab{Groupe de Spectrométrie Moléculaire et Atmosphérique (GSMA),\\
UMR CNRS 7331, UFR Sciences Exactes et Naturelles\\
Moulin de la Housse, Bt. 6, BP 1039\\
51687 Reims Cedex 2 - FRANCE
}
